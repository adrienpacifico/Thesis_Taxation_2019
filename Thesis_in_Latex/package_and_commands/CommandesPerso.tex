%%%%%%%%%%%%%%%%%%%%%%%%%%%%%%%%%%%%%%%%
%           Commandes perso            %
%%%%%%%%%%%%%%%%%%%%%%%%%%%%%%%%%%%%%%%%

\newcommand{\alp}{\texorpdfstring{\ensuremath{\upalpha}\xspace}{alpha }}
\newcommand{\bet}{\texorpdfstring{\ensuremath{\upbeta}\xspace}{b\'{e}ta }}
\newcommand{\alpbet}{\texorpdfstring{\ensuremath{\upalpha-\upbeta}\xspace}{alpha-b\'{e}ta}}
\newcommand{\alpt}{\ensuremath{\alpha_2}\xspace}
\newcommand{\strt}{\gls{strt}\xspace}


% Divers
\newcommand{\FRA}{$\text{FRA}$ }
\newcommand{\MR}{$\text{MR}$ }


% Tenseur des déformation cylindrique
\newcommand{\epsrr}{\ensuremath{\varepsilon_{rr}}\xspace}
\newcommand{\epstt}{\ensuremath{\varepsilon_{\theta\theta}}\xspace}
\newcommand{\epszz}{\ensuremath{\varepsilon_{zz}}\xspace}
\newcommand{\epsrt}{\ensuremath{\varepsilon_{r\theta}}\xspace}
\newcommand{\epstz}{\ensuremath{\varepsilon_{\theta z}}\xspace}
\newcommand{\epszr}{\ensuremath{\varepsilon_{zr}}\xspace}

\newcommand{\matlab}{\textsc{Matlab}\texttrademark\xspace}



%% Figures centrées, et en position 'here, top, bottom or page'
\newenvironment{figureth}{
		\begin{figure}[htbp]
			\centering
	}{
		\end{figure}
		}
		
		
%% Tableaux centrés, et en position 'here, top, bottom or page'
\newenvironment{tableth}{
		\begin{table}[htbp]
			\centering
			%\rowcolors{1}{coleurtableau}{coleurtableau}
	}{
		\end{table}
		}

%% Sous-figures centrées, en position 'top'		
\newenvironment{subfigureth}[1]{
	\begin{subfigure}[t]{#1}
	\centering
}{
	\end{subfigure}
}

\newcommand{\citationChap}[2]{
	\epigraph{\og \textit{#1} \fg{}}{#2}
}

% Macro pour créer une figure (avec notes optionnelles) dans le corps du texte %
\newenvironment{fig}[4][1]{
\begin{figure}[!ht]
	\begin{center}
	\begin{minipage}{14cm}
		\caption{#2}
	\end{minipage}
	\end{center}
	\begin{center}
	\begin{minipage}{#1}
	\begin{center}#3
	\end{center}
	\end{minipage}
	\if!#4!\empty \else \\
	\begin{scriptsize}
	\begin{minipage}{#1}\vspace{0.2cm} \par #4
	\end{minipage}
	\end{scriptsize} \fi }
	{\end{center}
\end{figure}}

% Macro d'inclusion de graphiques pdf %
\newcommand{\graphique}[2][1]{\begin{minipage}{\linewidth}\begin{center}\includegraphics[width=#1\linewidth,clip]{Figures/#2}\end{center}\end{minipage}}


% Macro pour créer un tableau (avec notes optionnelles) %
\newenvironment{tab}[4][1]
{\def\TMP{#3}\newdimen\TMPsize\settowidth{\TMPsize}{\TMP}
\begin{table}[!ht]
\begin{center}
\begin{minipage}{14cm}
\caption{#2}
\end{minipage}
\end{center}
\begin{center}
\begin{minipage}{#1}
\resizebox{\textwidth}{!}{#3}
\end{minipage}
\if!#4!\empty \else \\
\begin{scriptsize}
\begin{minipage}{#1}\vspace{0.2cm} \par #4
\end{minipage}
\end{scriptsize} \fi }
{\end{center}
\end{table}}

%%%%% Pour prendre en compte les commandes \tightlist issues des conversions makdown à Latex de pandoc
\providecommand{\tightlist}{%
  \setlength{\itemsep}{0pt}\setlength{\parskip}{0pt}}